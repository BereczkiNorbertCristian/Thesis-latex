\chapter{Fundamentarea teoretic�}
\label{CapFT}

Absolventul va prezenta detaliat (pe baza document�rii bibliografice \cite{Crn02}) problematica tratat�:
\begin{itemize}
	\item �ncadrarea temei �ntr-una mai general�;
	\item trecerea �n revist� a abord�rilor existente ale problemei cu marcarea avantajelor �i dezavantajelor;
	\item descompunerea �n subprobleme specifice �i prezentarea modului de rezolvare.
\end{itemize}

	Partea fundament�rii teoretice poate fi consituit� din mai multe capitole, ca de exemplu ``Stadiul actual din domeniu/State of art/Literature Review'' �i ``Modele teoretice �i metode folosite/Research Method'' �i ``Problem Statement''.

